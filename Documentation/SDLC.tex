\documentclass{article}
\usepackage{graphicx} % Required for inserting images

\title{Medication Tracker Application}
\author{Warren Lazarraga, Rhonda Elhaouli, Celine Chahine, Heba Sayed}
\date{September 2024}

\begin{document}

\maketitle

\section{Introduction}

\subsection{Goals and Objectives}
The primary objective of the [Medication Tracker Application] is to provide an interface for more efficient tracking of medication data between patients and physicians. A key feature of this product is the ability to monitor patient pill consumption and automate the replenishment process. The initial phase of the project will focus on building the foundational components of the application framework:
\begin{enumerate}
    \item Utilization of a \textbf{relational database} model with three main entities: Patient, Physician, and Medication.
    \item An \textbf{API server} will be utilized to streamline data transfer to the client side of the application.
    \item A \textbf{mobile client-side interface} will be implemented to facilitate patient data logging, with the additional goal of enhancing accessibility.
    \item A \textbf{client-side web-based interface} will be implemented to create and manage patient profiles and facilitate the handling of prescription information.

\end{enumerate}

\subsection{Scope}
The scope of this project encompasses the design and planning phases of the application development process, the implementation of client requirements, and integration and testing through comprehensive quality assurance management. As the project scales, the scope will extend to include ongoing maintenance, support, and additional testing to ensure compliance with regulatory requirements.

\section{Requirement Engineering}
\subsection{User Stories}
The user stories for this software capture the essential needs and experiences of patients and doctors as they manage natural medication intake and coordination. These stories outline key features such as medication scheduling, inventory tracking, and automated reordering, providing a clear view of how the software will support users in maintaining their health regimens. By focusing on real-world challenges, such as avoiding missed doses and optimizing shipment timing, these user stories ensure the software remains practical, user-friendly, and aligned with both patient and doctor requirements throughout the development process.
\begin{itemize}
    \item \textbf{US\_A}: There should be a patient profile and a doctor profile.
    \item \textbf{US\_B}: Doctors should be able to add dosage and medications to patient profiles.
    \item \textbf{US\_C}: Patients should be able to see medications scheduled, dosage, and quantity. 
    \item \textbf{US\_D}: Patients should be able to request more medication when running low. (does the person decide when to get notified?) [Doctor should be notified when patient medication is running low]
    \item \textbf{US\_E}: Patients should be able to add the start date if a new bottle and quantity are taken.
    \item \textbf{US\_F}: Patients should have as little input as possible. 
    \item \textbf{US\_G}: There should be a cost effective way to place orders.
    \item  \textbf{US\_H}: Notification system
    \item \textbf{US\_I}: Batch them up. Threshold of bulks in relation to shipping 
    \item \textbf{US\_J}: Ask only if you want to center around natural medication, such as supplements and vitamins. OTC?
    \item \textbf{US\_K}: The doctor should be able to place medication orders.
    \item \textbf{US\_L}: Patients should be able to see medications scheduled, dosage, and quantity. 
\end{itemize}
base on each of the user stories, the software requirements must include the following features: 
\begin{enumerate}
    \item \textbf{Pill Consumption Tracking:} Patients can log the type and number of pills they take based on their medication schedule, which may change a few times a month.
    \item \textbf{Inventory Monitoring:} The system tracks medication usage in real-time, alerting the patient when the supply is running low. It also factors in the estimated shipping time for new orders.
    \item \textbf{Automatic Reorder Requests:} The solution allows patients to send requests to their doctor for refilling prescriptions or ordering more supplements, ensuring there is no gap in their medication schedule. Medications are ordered in batches to minimize shipping costs.
    \item \textbf{Doctor-Patient Collaboration:} The system enables doctors to input the type of medications, dosage, and other relevant information, while patients are responsible for setting the start date of a new bottle and verifying that they’ve taken the prescribed dose.
\end{enumerate}
\subsection{Specificiation}
\section{Design and Planning}
\subsection{System Architecture}
\subsection{Data Flow }
\subsection{Entity Relationship Diagram}
\subsection{User Interfaces}


\end{document}
