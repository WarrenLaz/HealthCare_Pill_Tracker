\documentclass{article}
\usepackage{graphicx} % Required for inserting images

\title{Medication Tracker Application}
\author{Warren Lazarraga, Rhonda Elhaouli, Celine Chahine, Heba Sayed}
\date{September 2024}

\begin{document}

\maketitle

\section{Introduction}

\subsection{Goals and Objectives}
The primary objective of the [Medication Tracker Application] is to provide an interface for more efficient tracking of medication data between patients and physicians. A key feature of this product is the ability to monitor patient pill consumption and automate the replenishment process. The initial phase of the project will focus on building the foundational components of the application framework:
\begin{enumerate}
    \item Utilization of a \textbf{relational database} model with three main entities: Patient, Physician, and Medication.
    \item An \textbf{API server} will be utilized to streamline data transfer to the client side of the application.
    \item A \textbf{mobile client-side interface} will be implemented to facilitate patient data logging, with the additional goal of enhancing accessibility.
    \item A \textbf{client-side web-based interface} will be implemented to create and manage patient profiles and facilitate the handling of prescription information.

\end{enumerate}

\subsection{Scope}
The scope of this project encompasses the design and planning phases of the application development process, the implementation of client requirements, and integration and testing through comprehensive quality assurance management. As the project scales, the scope will extend to include ongoing maintenance, support, and additional testing to ensure compliance with regulatory requirements.

\section{Requirement Engineering}
\subsection{Requirements}
\subsubsection{Client Project Description}
I consume natural medications such as supplements to help balance the healthy vitamin and mineral levels in my body. I have a schedule of how many pills and the type of medication to take multiple times a day. I have to order the pills where it takes several days to arrive. I run out frequently because I overestimate the amount of pills I have and underestimate the time it takes for the pills to arrive. I need a way to make sure I don't miss any medication on my schedule. The schedule may change a couple times a month. The solution must account for the consumption and the time it takes for the pills to arrive. I have to order the medication manually today, but it would be nice to send a request to order more medication with my doctor so I never run out. The medication should be ordered in batches to minimize shipping costs. This solution should be shared with the doctor to add the medications and dosage while the patient can add the start date of a new bottle and verify the pill(s) were taken. Please let me know if more detail is required or if you have any questions. 
\begin{itemize}
    \item \textbf{US\_A}: There should be a patient profile and a doctor profile.
    \item \textbf{US\_B}: Doctors should be able to add dosage and medications to patient profiles.
    \item \textbf{US\_C}: Patients should be able to see medications scheduled, dosage, and quantity. 
    \item \textbf{US\_D}: Patients should be able to request more medication when running low. (does the person decide when to get notified?) [Doctor should be notified when patient medication is running low]
    \item \textbf{US\_E}: Patients should be able to add the start date if a new bottle and quantity are taken.
    \item \textbf{US\_F}: Patients should have as little input as possible. 
    \item \textbf{US\_G}: There should be a cost effective way to place orders.
    \item  \textbf{US\_H}: Notification system
    \item \textbf{US\_I}: Batch them up. Threshold of bulks in relation to shipping 
    \item \textbf{US\_J}: Ask only if you want to center around natural medication, such as supplements and vitamins. OTC?
    \item \textbf{US\_K}: The doctor should be able to place medication orders.
    \item \textbf{US\_L}: Patients should be able to see medications scheduled, dosage, and quantity. 

\end{itemize}


\end{document}
